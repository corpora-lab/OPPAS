\documentclass{article}
\usepackage[T1]{fontenc}
\usepackage[utf8]{inputenc}
\usepackage{libertine}
\usepackage{amsmath}
\usepackage{amssymb}
\usepackage{stmaryrd}
\usepackage{wasysym}
\usepackage{mathrsfs}
\usepackage{multirow}
\usepackage{rotating}
\usepackage[colorlinks]{hyperref}

%% Logical operators
% Next and Back
\newcommand*{\lnext}{\ocircle}
\newcommand*{\lanext}{\ocircle_{\chi}}
\newcommand*{\lgnext}[1]{\ocircle^{#1}}
%\newcommand*{\lganext}[1]{\ocircle_{\chi}^{#1}}
\newcommand*{\lganext}[1]{\chi_{F}^{#1}}
\newcommand*{\lback}{\circleddash}
\newcommand*{\laback}{\circleddash_{\chi}}
\newcommand*{\lgback}[1]{\circleddash^{#1}}
%\newcommand*{\lgaback}[1]{\circleddash_{\chi}^{#1}}
\newcommand*{\lgaback}[1]{\chi_{P}^{#1}}
\newcommand*{\lhynext}{\lgnext{\lessdot}_H}
\newcommand*{\lhyback}{\lgback{\lessdot}_H}
\newcommand*{\lhtnext}{\lgnext{\gtrdot}_H}
\newcommand*{\lhtback}{\lgback{\gtrdot}_H}
\newcommand*{\lthrnext}{\mathit{CallThr}}

% Until and Since
\newcommand*{\lguntil}[4]{#3 \mathbin{\mathcal{U}^{#1}_{#2}} #4}
\newcommand*{\luntil}[3]{#2 \mathbin{\mathcal{U}^{#1}} #3}
\newcommand*{\lluntil}[2]{\luntil{}{#1}{#2}}
\newcommand*{\lgsince}[4]{#3 \mathbin{\mathcal{S}^{#1}_{#2}} #4}
\newcommand*{\lsince}[3]{#2 \mathbin{\mathcal{S}^{#1}} #3}
\newcommand*{\llsince}[2]{\lsince{}{#1}{#2}}
\newcommand*{\lhyuntil}[2]{\lguntil{\lessdot}{H}{#1}{#2}}
\newcommand*{\lhysince}[2]{\lgsince{\lessdot}{H}{#1}{#2}}
\newcommand*{\lhtuntil}[2]{\lguntil{\gtrdot}{H}{#1}{#2}}
\newcommand*{\lhtsince}[2]{\lgsince{\gtrdot}{H}{#1}{#2}}
\newcommand*{\lcallsince}{\mathit{Scall}}
\newcommand*{\lfuntil}[3]{#2 \mathbin{\mathcal{U}(#1)} #3}

% Globally
\newcommand*{\lglob}[1]{\square^{#1}}
\newcommand*{\lpglob}[1]{\boxminus^{#1}}
\newcommand*{\llglob}{\square}
\newcommand*{\llpglob}{\boxminus}
\newcommand*{\lhyglob}{\lglob{\lessdot}_H}
\newcommand*{\lhypglob}{\lpglob{\lessdot}_H}
\newcommand*{\lhtglob}{\lglob{\gtrdot}_H}
\newcommand*{\lhuglob}{\lglob{u}_H}
\newcommand*{\lhupglob}{\lpglob{u}_H}

% Eventually
\newcommand*{\leven}[1]{\Diamond^{#1}}
\newcommand*{\lleven}{\Diamond}
\newcommand*{\lhyeven}{\leven{\uparrow}}
\newcommand*{\lhteven}{\leven{\downarrow}}

%% New POTL
\newcommand*{\ldnext}{\lnext^d}
\newcommand*{\lunext}{\lnext^u}
\newcommand*{\ldback}{\lback^d}
\newcommand*{\luback}{\lback^u}
\newcommand*{\lcnext}[1]{\chi_F^{#1}}
\newcommand*{\lcdnext}{\lcnext{d}}
\newcommand*{\lcunext}{\lcnext{u}}
\newcommand*{\lcback}[1]{\chi_P^{#1}}
\newcommand*{\lcdback}{\lcback{d}}
\newcommand*{\lcuback}{\lcback{u}}
\newcommand*{\lcduntil}[2]{{#1} \mathbin{\mathcal{U}_\chi^d} {#2}}
\newcommand*{\lcdsince}[2]{{#1} \mathbin{\mathcal{S}_\chi^d} {#2}}
\newcommand*{\lcuuntil}[2]{{#1} \mathbin{\mathcal{U}_\chi^u} {#2}}
\newcommand*{\lcusince}[2]{{#1} \mathbin{\mathcal{S}_\chi^u} {#2}}
\newcommand*{\lhnext}[1]{\lnext_H^{#1}}
\newcommand*{\lhback}[1]{\lback_H^{#1}}
\newcommand*{\lhdnext}{\lhnext{d}}
\newcommand*{\lhdback}{\lhback{d}}
\newcommand*{\lhduntil}[2]{{#1} \mathbin{\mathcal{U}_H^d} {#2}}
\newcommand*{\lhdsince}[2]{{#1} \mathbin{\mathcal{S}_H^d} {#2}}
\newcommand*{\lhunext}{\lhnext{u}}
\newcommand*{\lhuback}{\lhback{u}}
\newcommand*{\lhuuntil}[2]{{#1} \mathbin{\mathcal{U}_H^u} {#2}}
\newcommand*{\lhusince}[2]{{#1} \mathbin{\mathcal{S}_H^u} {#2}}

%% OP Relations
\newcommand*{\chain}{\chi}
\newcommand*{\fchain}{\overrightarrow{\chi}}
\newcommand*{\bchain}{\overleftarrow{\chi}}

%% Misc
\newcommand*{\powset}[1]{{\mathcal{P}(#1)}}
\newcommand*{\lsucc}{\operatorname{succ}}
\newcommand*{\prf}{\pi}
\newcommand*{\pr}{\mathrel{\prf}}

%% Model Checking
\newcommand*{\clos}[1]{\operatorname{Cl}({#1})}
\newcommand*{\closaux}[1]{\operatorname{Cl}_\mathit{aux}({#1})}
\newcommand*{\atoms}[1]{\operatorname{Atoms}({#1})}
\newcommand*{\fprop}[1]{\mathrm{q}_{(#1)}}

%% Translation to XPath
\newcommand*{\ra}{\operatorname{Rcc}}
\newcommand*{\invtau}{\tau^{-1}_{AP}}
\newcommand*{\first}{\operatorname{first}}
\newcommand*{\last}{\operatorname{last}}
\newcommand*{\rchild}{R_\Downarrow}
\newcommand*{\rsibl}{R_\Rightarrow}
\newcommand*{\rparen}{R_\Uparrow}
\newcommand*{\rlsibl}{R_\Leftarrow}
\newcommand*{\uotrees}{\mathcal{T}}
\newcommand*{\xuntil}{$\mathcal{X}_\mathit{until}$}

%% Example labels
\newcommand*{\lcall}{\mathbf{call}}
\newcommand*{\lret}{\mathbf{ret}}
\newcommand*{\lhandle}{\mathbf{han}}
\newcommand*{\lthrow}{\mathbf{exc}}
\newcommand*{\lstmt}{\mathbf{stm}}

%% Old stuff from old papers
\newcommand{\mrk}[1]{{#1}'}
\newcommand{\oldstack}[3]{%
{\ifthenelse{\equal{#1}{1}}{%
\mrk{#2}
}%
{#2}}_{#3}%
}
\newcommand{\stack}[3]{%
[%
{\ifthenelse{\equal{#1}{1}}{%
\mrk{#2}
}%
{#2}}
{\ifthenelse{\equal{#1}{0}}{\ }{} }
{#3}%
]%
}

\newcommand{\tstack}[2]{%
[#1,\ #2]%
}
\newcommand{\tconfig}[3]{\langle #1, \ #2, \ #3 \rangle}
\DeclareRobustCommand{\newvdash}{\shortmid\joinrel\relbar\joinrel\relbar}
\newcommand{\transition}[1]{\stackrel {{#1}} \vdash}
\newcommand{\powerset}{\mathcal{P}}

% per i cammini dell'automa
\newcommand{\va}[1]{\stackrel{#1}{\longrightarrow}}
\newcommand{\vshift}[1]{\stackrel{#1}{\dashrightarrow}}
\newcommand{\ourpath}[1]{\stackrel{#1}{\leadsto}}
\newcommand{\flush}[1]{\stackrel{#1}{\Longrightarrow}}

\newcommand{\nont}[4]{\langle^{#1} #2, #3 {}^{#4} \rangle}
\newcommand{\ochain}[3]{{}^{#1}[ #2 ]{}^{#3}}

%%% AUTOMA
\newcommand{\config}[3]{\langle #1, \allowbreak #2, \allowbreak #3 \rangle}
\newcommand{\symb}[1]{\mathop{smb}(#1)}
\newcommand{\state}[1]{\mathop{st}(#1)}
\newcommand{\markd}[1]{\mathop{marked}(#1)}
\newcommand{\tp}[1]{\mathop{top}(#1)}
\newcommand{\inpt}[1]{\mathop{input}(#1)}


%%% AUTOMA DETERMINIZZATO
\newcommand{\stpair}[2]{( {#1},{#2} )} % coppie di stati
\newcommand{\pair}[2]{\langle {#1},{#2}\rangle} % coppia (simbolo, set di stati)


\newcommand{\der}[1]{ \stackrel {{#1}} \Rightarrow}
\newcommand{\derh}[1]{ \stackrel {[{#1}]} \Rightarrow}
\newcommand{\comp}[1]{ \stackrel {{#1}} \vdash }
\newcommand{\comph}[1]{ \stackrel {[{#1}]} \vdash }



\title{POMC User's Guide}
\author{Michele Chiari}
\date{}

\begin{document}

\maketitle

\begin{abstract}
POMC is an implementation of the automaton construction for formulas
of Precedence-Oriented Temporal Logic (POTL), and the model checking
procedure thereof.
This document is a reference guide to its input and output formats,
and also describes at a high level its architecture and source code.
\end{abstract}

\section{Introduction}

Precedence-Oriented Temporal Logic (POTL) \cite{ChiariMP19} is a novel temporal
logic formalism based on the family of Operator Precedence Languages (OPL),
a subclass of deterministic context-free languages.
POTL is strictly more expressive than LTL and other temporal logics based
on subfamilies of context-free languages, such as CaRet \cite{AlurEM04}
and NWTL \cite{lmcs/AlurABEIL08}.
In particular, POTL reasons on an algebraic structure equipped with,
besides the usual linear order, a binary nesting relation between word positions,
which can be one-to-one, one-to-many, or many-to-one.
Such a relation is more general than the one found in Nested Words \cite{jacm/AlurM09},
because the latter may only be one-to-one.
POTL can be applied to the specification of several kinds of requirements
on procedural programs with exceptions.

Besides some results concerning its expressiveness,
we introduced an automata-based model checking procedure for POTL.
This procedure consists in building an Operator Precedence Automaton (OPA),
the class of pushdown automata that identifies OPL,
accepting the language denoted by a given POTL formula.
The size of the generated automaton is exponential in the length of the formula,
which is asymptotically comparable with other linear-time temporal logic
formalisms such as LTL, CaRet, and NWTL.

POMC is a tool that implements the automaton construction
for POTL, and a model checking procedure for it.
For the time being, only the construction for finite words has been implemented.
Given a POTL formula $\varphi$ and an input OPA modeling some system,
POMC builds the OPA equivalent to $\neg \varphi$, computes its intersection
with the input OPA, and checks the emptiness of the resulting OPA.
Both the OPA construction and the intersection are done on-the-fly.

We used POMC to prove some interesting properties of programs which we
modeled as OPA.  All such experiments are contained in
\texttt{\hyphenchar\font=`\-
  \hyphenpenalty=10000
  \exhyphenpenalty=-50
  pomc}
files in the \texttt{opa} and \texttt{opa-more} subdirectories.


If you wish to examine the input formulas and OPA to such experiments more carefully,
or to write your own, we describe the format of POMC input files in Section~\ref{sec:format}.
We also demonstrate the use of the tool with a few experiments in Section~\ref{sec:exp}.
Finally, Section~\ref{sec:sources} contains a high-level description of the source code.


\section{Quick-Start Guide}
\label{sec:quick-start}

POMC has been developed in the Haskell programming language,
and it has been packaged with the Haskell Tool Stack\footnote{\url{https://www.haskellstack.org/}}.
POMC can be built from sources by typing the following commands in a shell:
\begin{verbatim}
$ cd ~/path/to/POMC-sources
$ stack setup
$ stack build
\end{verbatim}
Then, POMC can be executed on an input file \verb|file.pomc| as follows:
\begin{verbatim}
$ stack exec pomc -- file.pomc
\end{verbatim}

Directory \verb|eval| contains several POMC input files. Such files
contain POTL formulas and OPA to be checked against them. For more
details on the format of POMC input files, see
Section~\ref{sec:format}.

Directory \verb|eval| also contains the Python script \verb|mcbench.py|,
which may be useful to evaluate POMC input files, as it also prints
a summary of the resources used by POMC.
It must be executed with a subdirectory of \verb|~/path/to/POMC-sources|
as its working directory.
If invoked with no arguments, it executes POMC on all input files in the current
working directory. E.g.,
\begin{verbatim}
$ cd ~/path/to/POMC-sources/eval
$ python3 mcbench.py opa
\end{verbatim}
evaluates all \verb|*.pomc| files in directory \verb|~/path/to/POMC-sources/eval/opa|.
The script can also be invoked with POMC files as its arguments,
which are then evaluated. E.g.,
\begin{verbatim}
$ cd ~/path/to/POMC-sources/eval/opa
$ python3 mcbench.py 1-generic-small.pomc
                                2-generic-medium.pomc
\end{verbatim}
executes POMC with input files \verb|1-generic-small.pomc| and \verb|2-generic-medium.pomc|.
\texttt{mcbench.py} can be invoked with the following optional flags:
\begin{description}
\item[\texttt{-iter <\#iters>}]
  Number of iterations of the benchmarks to be performed.
  The final table printed by the script contains the mean time
  and memory values computed on all iterations. (Default: 1)
\item[\texttt{-jobs <\#jobs>}]
  Number of benchmarks to be run in parallel.
  If you provide a value greater than 1, make sure you increase
  the number of cores available to the VM accordingly.
  (Default: 1)
\item[\texttt{-ms}]
  Output time in milliseconds.
\end{description}


\section{POMC Input/Output Format}
\label{sec:format}

\begin{figure}
\[
\begin{array}{r | c c c c}
         & \lcall   & \lret   & \lhandle & \lthrow \\
\hline
\lcall   & \lessdot & \doteq  & \lessdot & \gtrdot \\
\lret    & \gtrdot  & \gtrdot & \gtrdot & \gtrdot \\
\lhandle & \lessdot & \gtrdot & \lessdot & \doteq \\
\lthrow  & \gtrdot  & \gtrdot & \gtrdot  & \gtrdot \\
\end{array}
\]
\caption{OPM $M_\lcall$.}
\label{fig:opm}
\end{figure}

POMC takes in input plain text files of two possible formats.
The first one contains a requirement specification in terms of a list of
POTL formulas, and an OPA to be checked against them:
\begin{verbatim}
formulas = FORMULA [, FORMULA ...] ;
prec = SL PR SL [, SL PR SL ...] ;
opa:
  initials = STATE_SET ;
  finals = STATE_SET ;
  deltaPush = (STATE, AP_SET, STATE_SET)
                [, (STATE, AP_SET, STATE_SET) ...] ;
  deltaShift = (STATE, AP_SET, STATE_SET)
                [, (STATE, AP_SET, STATE_SET) ...] ;
  deltaPop = (STATE, STATE, STATE_SET)
                [, (STATE, STATE, STATE_SET) ...] ;
\end{verbatim}
where \texttt{STATE\_SET} is either a single state, or a space-separated list of states,
surrounded by parentheses.
States are non-negative integer numbers (e.g. \texttt{(0 1 ...)}).
\texttt{AP\_SET} is a space-separated list of atomic propositions, surrounded by parentheses
(e.g. \texttt{(call p1)} or \texttt{("call" "p1")})
In more detail:
\begin{itemize}
\item \texttt{prec}
  is followed by a comma-separated list of precedence relations between structural labels,
  that make up an Operator Precedence Matrix.
  The list is terminated by a semicolon.
  Precedence relations (\texttt{PR}) can be one of \texttt{<}, \texttt{=}, or \texttt{>},
  which respectively mean $\lessdot$, $\doteq$, and $\gtrdot$.
  Structural labels (\texttt{SL}) can be any sequence of alphabetic characters.
\item \texttt{formulas}
  is followed by a comma-separated, semicolon-terminated list of POTL formulas.
  The syntax of such formulas is defined later in this section.
\item \texttt{opa}
  is followed by the explicit description of an OPA.
  The list of initial and final states must be given, as well as the transition relations.
\end{itemize}

Additionally, POMC input files may contain C++-style single-line comments
starting with \verb|\\|, and C-style multi-line comments enclosed in
\verb|/*| and \verb|*/|.

External files can be included with
\begin{verbatim}
include = "path/to/file.inc";
\end{verbatim}
where the path is relative to the \texttt{pomc} file location.

POTL formulas can be written by using the operators in the ``POMC Operator'' column
of Table~\ref{tab:potl-syntax}, following the same syntax rules as in \cite{ChiariMP19}.

Once POMC is executed on an input file in the format above,
it checks whether the given OPA satisfies the given formulas,
one by one.

Consider the example input file \texttt{1-generic-small.pomc}, reported below:
\begin{verbatim}
prec = call < call, call = ret, call < han, call > exc,
       ret > call,  ret > ret,  ret > han,  ret > exc,
       han < call,  han > ret,  han < han,  han = exc,
       exc > call,  exc > ret,  exc > han,  exc > exc;

formulas = G ((call And pb And (T Sd (call And pa))) --> (PNu exc Or XNu exc));

opa:
  initials = 0;
  finals = 10;
  deltaPush =
    (0, (call pa),   1),
    (1, (han),       2),
    (2, (call pb),   3),
    (3, (call pc),   4),
    (4, (call pc),   4),
    (6, (call perr), 7),
    (8, (call perr), 7);
  deltaShift =
    (4, (exc),       5),
    (7, (ret perr),  7),
    (9, (ret pa),    11);
  deltaPop =
    (4, 2, 4),
    (4, 3, 4),
    (4, 4, 4),
    (5, 1, 6),
    (7, 6, 8),
    (7, 8, 9),
    (11, 0, 10);
\end{verbatim}
First, OPM $M_\lcall$ from \cite{ChiariMP19} (Figure~\ref{fig:opm}) is chosen.

The meaning of the formula
\texttt{G ((call And pb And (T Sd (call And pa))) --> (PNu exc Or XNu exc))},
or
\(
  \llglob \big((\lcall \land \mathrm{p}_B \land
    \lcallsince(\top, \mathrm{p}_A))
    \implies \lthrnext(\top) \big)
\),
is explained in the paper.

POMC will check the OPA against the formula, yielding the following output:
\begin{verbatim}
_________________________________________________________
Evaluating file opa/1-generic-small.pomc ...

Model Checking
Formula: G ((("call" And "pb") And (T Sd ("call" And "pa")))
                --> ((PNu "exc") Or (XNu "exc")))
Input OPA state count: 12
Result:  True
Elapsed time: 14.59 s


Total elapsed time: 14.59 s (1.4593e1 s)

Max memory used (KB): 129152
\end{verbatim}
Indeed, the OPA does satisfy the formula.
POMC additionally outputs the time taken by each acceptance check,
and the maximum resident memory used.

\begin{table}
\centering
\begin{tabular}{| c | l | l | l | l |}
\hline
Group & POTL Operator & POMC Operator & Notation & Associativity \\
\hline
\hline
\multirow{15}{*}{\rotatebox[origin=c]{90}{Unary}}
& $\neg$ & \verb!~!, \texttt{Not} & Prefix & -- \\
& $\ldnext$ & \texttt{PNd}           & Prefix & -- \\
& $\lunext$ & \texttt{PNu}           & Prefix & -- \\
& $\ldback$ & \texttt{PBd}           & Prefix & -- \\
& $\luback$ & \texttt{PBu}           & Prefix & -- \\
& $\lcdnext$ & \texttt{XNd}           & Prefix & -- \\
& $\lcunext$ & \texttt{XNu}           & Prefix & -- \\
& $\lcdback$ & \texttt{XBd}           & Prefix & -- \\
& $\lcuback$ & \texttt{XBu}           & Prefix & -- \\
& $\lhdnext$ & \texttt{HNd}           & Prefix & -- \\
& $\lhunext$ & \texttt{HNu}           & Prefix & -- \\
& $\lhdback$ & \texttt{HBd}           & Prefix & -- \\
& $\lhuback$ & \texttt{HBu}           & Prefix & -- \\
& $\lleven$ & \texttt{F}, \texttt{Eventually} & Prefix & -- \\
& $\llglob$ & \texttt{G}, \texttt{Always}     & Prefix & -- \\
\hline
\multirow{8}{*}{\rotatebox[origin=c]{90}{POTL Binary}}
& $\lcduntil{}{}$ & \texttt{Ud}            & Infix  & Right \\
& $\lcuuntil{}{}$ & \texttt{Uu}            & Infix  & Right \\
& $\lcdsince{}{}$ & \texttt{Sd}            & Infix  & Right \\
& $\lcusince{}{}$ & \texttt{Su}            & Infix  & Right \\
& $\lhduntil{}{}$ & \texttt{HUd}           & Infix  & Right \\
& $\lhuuntil{}{}$ & \texttt{HUu}           & Infix  & Right \\
& $\lhdsince{}{}$ & \texttt{HSd}           & Infix  & Right \\
& $\lhusince{}{}$ & \texttt{HSu}           & Infix  & Right \\
\hline
\multirow{5}{*}{\rotatebox[origin=c]{90}{Prop.\ Binary}}
& $\land$ & \texttt{And}, \verb!&&! & Infix  & Left \\
& $\lor$ & \texttt{Or}, \verb!||!  & Infix  & Left \\
& $\oplus$ & \texttt{Xor}           & Infix  & Left \\
& $\implies$ & \texttt{Implies}, \verb!-->!  & Infix  & Right \\
& $\iff$ & \texttt{Iff}, \verb!<-->!     & Infix  & Right \\
\hline
\end{tabular}
\caption{This table contains all currently supported POTL operators, in descending order
of precedence. Operators listed on the same line are synonyms. Operators in the
same group have the same precedence. Note that operators are case sensitive.}
\label{tab:potl-syntax}
\end{table}


The second kind of input files also contains POTL formulas,
and a program in the \emph{MiniProc} language to be checked against them.
MiniProc is a simplified procedural programming language, where variables are all global
and only take Boolean values (note that MiniProc is not Turing-complete,
so any use of the word `program' when referring to it is a deliberate abuse of terminology).
This limitation allows POMC to translate every MiniProc program into an OPA,
that is then checked against the supplied formulas.
This kind of input files have this form:
\begin{verbatim}
formulas = FORMULA [, FORMULA ...] ;
program:
PROGRAM
\end{verbatim}
MiniProc programs have the following syntax:
\begin{verbatim}
PROGRAM := [DECLS]; FUNCTION [FUNCTION ...]
DECLS := var IDENTIFIER [, IDENTIFIER ...] ;
FUNCTION := IDENTIFIER () { STMT; [STMT ...] }
STMT := IDENTIFIER := BEXPR
      | while (BEXPR) { [STMT ...] }
      | if (BEXPR) { [STMT ...] } else { [STMT ...] }
      | try { [STMT ...] } catch { [STMT ...] }
      | IDENTIFIER()
      | throw
BEXPR := BEXPR && BDISJ | BDISJ
BDISJ := BDISJ || BTERM | BTERM
BTERM := !BTERM | (BEXPR) | IDENTIFIER | true | false
\end{verbatim}
In the definition, non-terminal symbols are uppercase, and keywords lowercase.
Parts surrounded by square brackets are optional, and ellipses mean that the enclosing
group can be repeated zero or more times.
An \texttt{IDENTIFIER} is any sequence of letters, numbers, or characters \texttt{.} and \texttt{\_},
starting with a letter or an underscore.
The program starts with a variable declaration, which mist include all variables used in the program.
Then, a sequence of functions are defined, the first one being the entry-point to the program.
Function bodies consist of semicolon-separated statements.
Assignments, while loops and ifs have the usual semantics.
The try-catch statement executes the catch block whenever an exception is thrown by any statement
in the try block (or any function it calls).
Exceptions are thrown by the \texttt{throw} statement, and they are not typed
(i.e., there is no way to distinguish different kinds of exceptions).
Functions can be called by prepending their name to the \texttt{()} token
(they do not admit arguments, as all variables are global).
Since all variables are Boolean, expressions can be composed with the logical and (\texttt{\&\&}),
or (\texttt{||}) and negation (\texttt{!}) operators.

An example input file is given below:
\begin{verbatim}
formulas = G ((call And pb And (call Sd (call And pa))) --> (PNu exc Or XNu exc));

program:
var foo;

pa() {
  foo = false;
  try {
    pb();
  } catch {
    pc();
  }
}

pb() {
  if (foo) {
    throw;
  } else {}
}

pc() { }
\end{verbatim}


\clearpage

\section{Some experiments}
\label{sec:exp}

In this section we report the results of some experiments, that
are provided in the \texttt{eval} directory.


\subsection{Directory \texttt{opa}}

This directory contains a few programs modeled as OPA,
on which POMC proves or disproves some interesting specifications.
The resources employed by POMC on such tasks are reported in Table~\ref{tab:eval}.
If you wish to repeat such experiments, you may run the following
commands:
\begin{verbatim}
$ cd ~/artifact/contents
$ python3 mcbench.py opa
\end{verbatim}

\begin{table}[b]
  \centering
  \begin{tabular}{| r | l | r | r | r | l |}
    \hline
    & Benchmark name & \# states & Time (s) & Mem. (MB) & Result \\
    \hline
    1 & generic small & 12 & 14 & 126 & True \\
    2 & generic medium & 24 & 15 & 119 & False \\
    3 & generic larger & 30 & 16 & 233 & True \\
    4 & Jensen & 42 & 1 & 70 & True \\
    5 & unsafe stack & 63 & 46 & 185 & False \\
    6 & safe stack & 77 & 44 & 70 & True \\
    7 & unsafe stack neutrality & 63 & 62 & 1,787 & True \\
    8 & safe stack neutrality & 77 & 29 & 126 & True \\
    \hline
  \end{tabular}
  \caption{Results of the evaluation.}
  \label{tab:eval}
\end{table}

\paragraph{Generic procedural programs.}
Formula
\[
  \llglob \big((\lcall \land \mathrm{p}_B \land
    \lcallsince(\top, \mathrm{p}_A))
    \implies \lthrnext(\top) \big)
\]
means that whenever procedure $\mathrm{p}_B$ is executed and at least
one instance of $\mathrm{p}_A$ is on the stack, $\mathrm{p}_B$ is
terminated by an exception.  We checked it against three OPA
representing some simple procedural programs with exceptions and
recursive procedures.  The formula holds on benchmarks no.\ 1 and 3,
but not on no.\ 2.

\paragraph{Stack Inspection.}
\cite{JensenLT99} contains an example Java program for managing a bank
account, which uses the security framework of the Java Development Kit
to enforce user permissions. The program allows the user to check the
account balance, and to withdraw money. To perform such tasks, the
invoking program must have been granted permissions \texttt{CanPay}
and \texttt{Debit}, respectively.  We modeled such program as an OPA
(bench.\ 4), and proved that the program enforces such security
measures effectively by checking it against the formula
\[
\llglob (\lcall \land \mathtt{read} \implies
  \neg (\lcdsince{\top}{(\lcall \land
                         \neg \mathtt{CanPay}
                         \land \neg \mathtt{read})}))
\]
meaning that the account balance cannot be read if some function in
the stack lacks the \texttt{CanPay} permission (a similar formula
checks the \texttt{Debit} permission).

\paragraph{Exception Safety.}
\cite{Sutter97} is a tutorial on how to make exception-safe generic
containers in C++. It presents two implementations of a generic stack
data structure, parametric on the element type \texttt{T}. The first
one is not exception-safe: if the constructor of \texttt{T} throws an
exception during a pop action, the topmost element is removed, but it
is not returned, and it is lost.  This violates the strong exception
safety \cite{Abrahams00} requirement that each operation is rolled
back if an exception is thrown. The second version of the data
structure instead satisfies such requirement.

While exception safety is, in general, undecidable, it is possible to
prove the stronger requirement that each modification to the data
structure is only committed once no more exceptions can be thrown.
We modeled both versions as OPA, and checked
such requirement with the following formula:
\[
\llglob (\lthrow \implies
         \neg ((\luback \mathtt{modified} \lor
               \lcuback \mathtt{modified})
         \land \lcuback (\mathtt{Stack::push} \lor \mathtt{Stack::pop})))
\]
POMC successfully found a counterexample for the first implementation (5),
and proved the safety of the second one (6).

Additionally, we proved that both implementations are
\emph{exception neutral} (7, 8), i.e.\ they do not block exceptions
thrown by the underlying types.


\subsection{Directory \texttt{opa-more}}

\begin{table}
\centering
\begin{tabular}{| l | c |}
\hline
Formula & Time (s) \\
\hline
\hline
$\lcdnext \mathrm{p}_\mathit{err}$ & 0.043 \\
$\ldnext (\ldnext (\lcall \land \lcunext \lthrow))$ & 0.094 \\
$\ldnext (\lhandle \land (\lcdnext (\lthrow \land \lcuback \lcall)))$ & 0.889 \\
$\llglob (\lthrow \implies \lcuback \lcall)$ & 0.051 \\
$\lcduntil{\top}{\lthrow}$ & 0.006 \\
$\ldnext (\ldnext (\lcduntil{\top}{\lthrow}))$ & 0.021 \\
$\llglob ((\lcall \land \mathrm{p}_A \land (\lcduntil{\neg \lret}{\mathrm{WRx}})) \implies \lcunext \lthrow)$ & 214 \\
$\ldnext (\lunext \lcall)$ & 0.001 \\
$\ldnext (\ldnext (\ldnext (\luback \lcall)))$ & 0.003 \\
$\lcdnext (\ldnext (\luback \lcall))$ & 0.081 \\
$\llglob ((\lcall \land \mathrm{p}_A \land (\lunext \lthrow \lor \lcunext \lthrow)) \implies (\lunext \mathrm{e}_B \lor \lcunext \mathrm{e}_B))$ & 541 \\
$\lleven (\lhdnext \mathrm{p}_B)$ & 0.053 \\
$\lleven (\lhdback \mathrm{p}_B)$ & 0.010 \\
$\lleven (\mathrm{p}_A \land (\lhduntil{\lcall}{\mathrm{p}_C}))$ & 0.36 \\
$\lleven (\mathrm{p}_C \land (\lhdsince{\lcall}{\mathrm{p}_A}))$ & 0.252 \\
$\llglob ((\mathrm{p}_C \land \lcunext \lthrow) \implies (\lhdsince{\neg \mathrm{p}_A}{\mathrm{p}_B}))$ & 567 \\
$\llglob (\lcall \land \mathrm{p}_B \implies \lhuuntil{\neg \mathrm{p}_C}{\mathrm{p}_\mathit{err}})$ & 0.137 \\
$\lleven (\lhunext \mathrm{p}_\mathit{err})$ & 0.003 \\
$\lleven (\lhuback \mathrm{p}_\mathit{err})$ & 0.002 \\
$\lleven (\mathrm{p}_A \land (\lhuuntil{\lcall}{\mathrm{p}_B}))$ & 0.032 \\
$\lleven (\mathrm{p}_B \land (\lhusince{\lcall}{\mathrm{p}_A}))$ & 0.031 \\
$\ldnext \lcall$ & 0.001 \\
$\ldnext (\ldnext \lcall)$ & 0.001 \\
$\ldnext (\ldnext (\ldnext \lcall))$ & 0.001 \\
$\ldnext (\ldnext (\ldnext (\ldnext \lcall)))$ & 0.004 \\
$\llglob (\lcall \implies \lcdnext \lret)$ & 0.095 \\
$\llglob (\lcall \implies \neg \lunext \lthrow)$ & 0.003 \\
$\llglob (\lcall \land \mathrm{p}_A \implies \neg (\lunext \lthrow \lor \lcunext \lthrow))$ & 0.647 \\
$\llglob (\lthrow \implies \neg (\luback (\lcall \land \mathrm{p}_A) \lor \lcuback (\lcall \land \mathrm{p}_A)))$ & 0.676 \\
$\llglob ((\lcall \land \mathrm{p}_B \land (\lcdsince{\lcall}{(\lcall \land \mathrm{p}_A)})) \implies \lunext \lthrow \lor \lcunext \lthrow$ & 32.4 \\
$\llglob (\lhandle \implies \lcunext \lret)$ & 0.090 \\
$\lcuuntil{\top}{\lthrow}$ & 0.045 \\
$\ldnext (\ldnext (\lcuuntil{\top}{\lthrow}))$ & 0.205 \\
$\ldnext (\ldnext (\ldnext (\lcuuntil{\top}{\lthrow})))$ & 0.694 \\
$\llglob (\lcall \land \mathrm{p}_C \implies (\lcuuntil{\top}{\lthrow \land \lcdback \lhandle}))$ & 5.06 \\
$\lcduntil{\lcall}{(\lret \land \mathrm{p}_\mathit{err})}$ & 0.145 \\
$\lcdnext (\lcall \land (\lcusince{(\lcall \lor \lthrow)}{\mathrm{p}_B}))$ & 1.44 \\
$\ldnext (\ldnext (\lcuuntil{(\lcall \lor \lthrow)}{\lret}))$ & 0.332 \\
\hline
Total & 1366 \\
\hline
\hline
Maximum memory used (MB) & 18,956.048 \\
\hline
\end{tabular}
\caption{Results of the additional experiments on OPA ``generic small''.}
\label{tab:more-exp-1}
\end{table}

\begin{table}
\centering
\begin{tabular}{| l | c |}
\hline
Formula & Time (s) \\
\hline
\hline
$\lcdnext \mathrm{p}_\mathit{err}$ & 0.006 \\
$\ldnext (\ldnext (\lcall \land \lcunext \lthrow))$ & 0.543 \\
$\ldnext (\lhandle \land (\lcdnext (\lthrow \land \lcuback \lcall)))$ & 3.24 \\
$\llglob (\lthrow \implies \lcuback \lcall)$ & 0.055 \\
$\lcduntil{\top}{\lthrow}$ & 0.017 \\
$\ldnext (\ldnext (\lcduntil{\top}{\lthrow}))$ & 0.041 \\
$\llglob ((\lcall \land \mathrm{p}_A \land (\lcduntil{\neg \lret}{\mathrm{WRx}})) \implies \lcunext \lthrow)$ & 616 \\
$\ldnext (\lunext \lcall)$ & 0.001 \\
$\ldnext (\ldnext (\ldnext (\luback \lcall)))$ & 0.009 \\
$\lcdnext (\ldnext (\luback \lcall))$ & 0.008 \\
$\llglob ((\lcall \land \mathrm{p}_A \land (\lunext \lthrow \lor \lcunext \lthrow)) \implies (\lunext \mathrm{e}_B \lor \lcunext \mathrm{e}_B))$ & 673 \\
$\lleven (\lhdnext \mathrm{p}_B)$ & 0.005 \\
$\lleven (\lhdback \mathrm{p}_B)$ & 0.007 \\
$\lleven (\mathrm{p}_A \land (\lhduntil{\lcall}{\mathrm{p}_C}))$ & 1.29 \\
$\lleven (\mathrm{p}_C \land (\lhdsince{\lcall}{\mathrm{p}_A}))$ & 1.48 \\
$\llglob ((\mathrm{p}_C \land \lcunext \lthrow) \implies (\lhdsince{\neg \mathrm{p}_A}{\mathrm{p}_B}))$ & 2253 \\
$\llglob (\lcall \land \mathrm{p}_B \implies \lhuuntil{\neg \mathrm{p}_C}{\mathrm{p}_\mathit{err}})$ & 0.310 \\
$\lleven (\lhunext \mathrm{p}_\mathit{err})$ & 0.002 \\
$\lleven (\lhuback \mathrm{p}_\mathit{err})$ & 0.001 \\
$\lleven (\mathrm{p}_A \land (\lhuuntil{\lcall}{\mathrm{p}_B}))$ & 0.035 \\
$\lleven (\mathrm{p}_B \land (\lhusince{\lcall}{\mathrm{p}_A}))$ & 0.028 \\
$\ldnext \lcall$ & 0.001 \\
$\ldnext (\ldnext \lcall)$ & 0.001 \\
$\ldnext (\ldnext (\ldnext \lcall))$ & 0.001 \\
$\ldnext (\ldnext (\ldnext (\ldnext \lcall)))$ & 0.003 \\
$\llglob (\lcall \implies \lcdnext \lret)$ & 0.037 \\
$\llglob (\lcall \implies \neg \lunext \lthrow)$ & 0.004 \\
$\llglob (\lcall \land \mathrm{p}_A \implies \neg (\lunext \lthrow \lor \lcunext \lthrow))$ & 0.367 \\
$\llglob (\lthrow \implies \neg (\luback (\lcall \land \mathrm{p}_A) \lor \lcuback (\lcall \land \mathrm{p}_A)))$ & 0.622 \\
$\llglob ((\lcall \land \mathrm{p}_B \land (\lcdsince{\lcall}{(\lcall \land \mathrm{p}_A)})) \implies \lunext \lthrow \lor \lcunext \lthrow$ & 0.046 \\
$\llglob (\lhandle \implies \lcunext \lret)$ & 0.152 \\
$\lcuuntil{\top}{\lthrow}$ & 0.215 \\
$\ldnext (\ldnext (\lcuuntil{\top}{\lthrow}))$ & 1.30 \\
$\ldnext (\ldnext (\ldnext (\lcuuntil{\top}{\lthrow})))$ & 3.79 \\
$\llglob (\lcall \land \mathrm{p}_C \implies (\lcuuntil{\top}{\lthrow \land \lcdback \lhandle}))$ & 5.84 \\
$\lcduntil{\lcall}{(\lret \land \mathrm{p}_\mathit{err})}$ & 0.014 \\
$\lcdnext (\lcall \land (\lcusince{(\lcall \lor \lthrow)}{\mathrm{p}_B}))$ & 0.856 \\
$\ldnext (\ldnext (\lcuuntil{(\lcall \lor \lthrow)}{\lret}))$ & 0.053 \\
\hline
Total & 3607 \\
\hline
\hline
Maximum memory used (MB) & 94,535.576 \\
\hline
\end{tabular}
\caption{Results of the additional experiments on OPA ``generic medium''.}
\label{tab:more-exp-2}
\end{table}

\begin{table}
\centering
\begin{tabular}{| l | c |}
\hline
Formula & Time (s) \\
\hline
\hline
$\lcdnext \mathrm{p}_\mathit{err}$ & 0.003 \\
$\ldnext (\ldnext (\lcall \land \lcunext \lthrow))$ & 0.517 \\
$\ldnext (\lhandle \land (\lcdnext (\lthrow \land \lcuback \lcall)))$ & 0.243 \\
$\llglob (\lthrow \implies \lcuback \lcall)$ & 0.054 \\
$\lcduntil{\top}{\lthrow}$ & 0.008 \\
$\ldnext (\ldnext (\lcduntil{\top}{\lthrow}))$ & 0.018 \\
$\llglob ((\lcall \land \mathrm{p}_A \land (\lcduntil{\neg \lret}{\mathrm{WRx}})) \implies \lcunext \lthrow)$ & 728 \\
$\ldnext (\lunext \lcall)$ & 0.001 \\
$\ldnext (\ldnext (\ldnext (\luback \lcall)))$ & 0.001 \\
$\lcdnext (\ldnext (\luback \lcall))$ & 0.008 \\
$\llglob ((\lcall \land \mathrm{p}_A \land (\lunext \lthrow \lor \lcunext \lthrow)) \implies (\lunext \mathrm{e}_B \lor \lcunext \mathrm{e}_B))$ &  1843 \\
$\lleven (\lhdnext \mathrm{p}_B)$ & 0.016 \\
$\lleven (\lhdback \mathrm{p}_B)$ & 0.006 \\
$\lleven (\mathrm{p}_A \land (\lhduntil{\lcall}{\mathrm{p}_C}))$ & 0.446 \\
$\lleven (\mathrm{p}_C \land (\lhdsince{\lcall}{\mathrm{p}_A}))$ & 0.466 \\
$\llglob ((\mathrm{p}_C \land \lcunext \lthrow) \implies (\lhdsince{\neg \mathrm{p}_A}{\mathrm{p}_B}))$ & 1764 \\
$\llglob (\lcall \land \mathrm{p}_B \implies \lhuuntil{\neg \mathrm{p}_C}{\mathrm{p}_\mathit{err}})$ & 0.132 \\
$\lleven (\lhunext \mathrm{p}_\mathit{err})$ & 0.002 \\
$\lleven (\lhuback \mathrm{p}_\mathit{err})$ & 0.001 \\
$\lleven (\mathrm{p}_A \land (\lhuuntil{\lcall}{\mathrm{p}_B}))$ & 0.028 \\
$\lleven (\mathrm{p}_B \land (\lhusince{\lcall}{\mathrm{p}_A}))$ & 0.027 \\
$\ldnext \lcall$ & 0.001 \\
$\ldnext (\ldnext \lcall)$ & 0.002 \\
$\ldnext (\ldnext (\ldnext \lcall))$ & 0.001 \\
$\ldnext (\ldnext (\ldnext (\ldnext \lcall)))$ & 0.008 \\
$\llglob (\lcall \implies \lcdnext \lret)$ & 0.030 \\
$\llglob (\lcall \implies \neg \lunext \lthrow)$ & 0.003 \\
$\llglob (\lcall \land \mathrm{p}_A \implies \neg (\lunext \lthrow \lor \lcunext \lthrow))$ & 0.743 \\
$\llglob (\lthrow \implies \neg (\luback (\lcall \land \mathrm{p}_A) \lor \lcuback (\lcall \land \mathrm{p}_A)))$ & 0.622 \\
$\llglob ((\lcall \land \mathrm{p}_B \land (\lcdsince{\lcall}{(\lcall \land \mathrm{p}_A)})) \implies \lunext \lthrow \lor \lcunext \lthrow$ & 31.9 \\
$\llglob (\lhandle \implies \lcunext \lret)$ & 0.362 \\
$\lcuuntil{\top}{\lthrow}$ & 0.241 \\
$\ldnext (\ldnext (\lcuuntil{\top}{\lthrow}))$ & 1.39 \\
$\ldnext (\ldnext (\ldnext (\lcuuntil{\top}{\lthrow})))$ & 4.80 \\
$\llglob (\lcall \land \mathrm{p}_C \implies (\lcuuntil{\top}{\lthrow \land \lcdback \lhandle}))$ & 5.78 \\
$\lcduntil{\lcall}{(\lret \land \mathrm{p}_\mathit{err})}$ & 0.027 \\
$\lcdnext (\lcall \land (\lcusince{(\lcall \lor \lthrow)}{\mathrm{p}_B}))$ & 0.728 \\
$\ldnext (\ldnext (\lcuuntil{(\lcall \lor \lthrow)}{\lret}))$ & 1.85 \\
\hline
Total & 4385 \\
\hline
\hline
Maximum memory used (MB) & 60,318.568 \\
\hline
\end{tabular}
\caption{Results of the additional experiments on OPA ``generic larger''.}
\label{tab:more-exp-3}
\end{table}

This directory contains more experiments devised with the purpose of
testing all POTL operators, also in order to find the most critical
cases.  In fact, the complexity of POTL model checking is exponential
in the length of the formula.  This is of course unsurprising, since
it subsumes logics such as LTL and NWTL, whose model checking is also
exponential.  Actually, model checking is feasible for many
specifications useful in practice.  There are, however, some cases in
which the exponentiality of the construction becomes evident.

In Tables~\ref{tab:more-exp-1}, \ref{tab:more-exp-2} and
\ref{tab:more-exp-3} we show the results of model checking numerous
POTL formulas on three OPA representing generic procedural programs.
Some of them are checked very quickly,
while others require a long execution time and a very large amount of
memory.  Because of the memory requirements of some of such formulas,
these results were run on a server with a 2.0 GHz 16-core AMD CPU and
500 GB of RAM (the fact that the CPU has 16 cores is actually
irrelevant, as POMC only runs on one for the time being).

If you wish to repeat such experiments, you may run the following
commands:
\begin{verbatim}
$ cd ~/artifact/contents
$ python3 mcbench.py opa-more
\end{verbatim}
Of course, a machine with an appropriate amount of RAM is needed.


\section{Source Code}
\label{sec:sources}

The source code of POMC is contained in the \texttt{src/Pomc} directory.
We describe the contents of each file below.
\begin{description}
\item[Check.hs] This file contains the data structures and functions
  that implement the translation of POTL formulas into OPA.
  The \texttt{check} and \texttt{fastcheck} functions build the OPA
  and check for string acceptance. \texttt{makeOpa} returns
  a thunk containing an un-evaluated OPA, which is built on-the-fly
  while the calling context evaluates the transition functions.
\item[Encoding.hs] contains a data structure that represents a set of POTL
  formulas as a bit vector. It is use to encode OPA states in a
  memory-efficient form in Check.hs.
\item[MiniProc.hs] contains code that translates MiniProc programs into OPA.
\item[MiniProcParse.hs] contains a parser for MiniProc programs.
\item[ModelChecker.hs] contains the model checking launcher functions,
  and a data structure to represent the input OPA to be checked explicitly.
  It calls \texttt{makeOpa} to translate the negation of the specification
  into an equivalent OPA, creates a thunk representing an un-evaluated
  intersection of the two OPA, and then uses the reachability algorithm from
  Satisfiability.hs to determine emptiness.
\item[Opa.hs] contains an implementation of OPA, which is used to test
  string acceptance.
\item[OpaGen.hs] contains a simple automated OPA generator (still experimental).
\item[Parse.hs] contains a parser for POMC input files.
\item[Potl.hs] defines the datatype for POTL formulas.
\item[Prec.hs] defines the data type for precedence relations.
\item[Prop.hs] defines the data type for atomic propositions.
\item[PropConv.hs] contains dome functions useful to change the representation
  of atomic propositions from strings to unsigned integers.
  This is used by other parts of the program to achieve better performances,
  as strings are represented as lists of char in Haskell, which is quite
  inefficient.
\item[Satisfiability.hs] contains the reachability algorithms used in the
  model checker to decide OPA emptiness. They can also be use to decide
  satisfiability of a formula.
\item[Util.hs] contains various functions used in other parts of the code.
\end{description}


\bibliographystyle{abbrv}
\bibliography{optl}

\end{document}
